%%%%%%%%%%%%%%%%%%%%%%%%%%%%%%%%%%%%%%%%%%%%%%%%%%%%%%%%%%%%
%%%%%%%%%%%%%%%%%%%%%%%%%%%%%%%%%%%%%%%%%%%%%%%%%%%%%%%%%%%%
%%%%%%%%%%%%%%%%%%%%%%%%%%%%%%%%%%%%%%%%%%%%%%%%%%%%%%%%%%%%
%%%%%%%%%%%%%%%%%%%%%%%%%%%%%%%%%%%%%%%%%%%%%%%%%%%%%%%%%%%%
%%%%%%%%%%%%%%%%%%%%%%%%%%%%%%%%%%%%%%%%%%%%%%%%%%%%%%%%%%%%

\documentclass[12pt]{article}
\usepackage{fancyhdr}
\usepackage{pslatex}
\usepackage{epsfig}
\usepackage{times}
\usepackage{amsmath}
\usepackage{mathrsfs}
\usepackage[dvipsnames]{xcolor}
\usepackage[hidelinks]{hyperref}%renewcommand{\topfraction}{1.0}
\renewcommand{\topfraction}{1.0}
\renewcommand{\bottomfraction}{1.0}
\renewcommand{\textfraction}{0.0}
\setlength {\textwidth}{6.6in}
\hoffset=-1.0in
\oddsidemargin=1.00in
\marginparsep=0.0in
\marginparwidth=0.0in                                                                               
\setlength {\textheight}{9.0in}
\voffset=-1.00in
\topmargin=1.0in
\headheight=0.0in
\headsep=0.00in
\footskip=0.50in                                         
\setcounter{page}{1}
\begin{document}
\def\pos{\medskip\quad}
\def\subpos{\smallskip \qquad}
\newfont{\nice}{cmr12 scaled 1250}
\newfont{\name}{cmr12 scaled 1080}
\newfont{\swell}{cmbx12 scaled 800}
%%%%%%%%%%%%%%%%%%%%%%%%%%%%%%%%%%%%%%%%%%%%%%%%%%%%%%%%%%%%
%     DO NOT CHANGE ANYTHING ABOVE THIS LINE
%%%%%%%%%%%%%%%%%%%%%%%%%%%%%%%%%%%%%%%%%%%%%%%%%%%%%%%%%%%%
%     DO NOT CHANGE ANYTHING ABOVE THIS LINE
%%%%%%%%%%%%%%%%%%%%%%%%%%%%%%%%%%%%%%%%%%%%%%%%%%%%%%%%%%%%
%     DO NOT CHANGE ANYTHING ABOVE THIS LINE
%%%%%%%%%%%%%%%%%%%%%%%%%%%%%%%%%%%%%%%%%%%%%%%%%%%%%%%%%%%%

\begin{center}
{\large
PHYS 20323/60323: Fall 2025 - LaTeX Example
}\\
%%%%%%%%%%%%%%%%%%%%%%%%%%%%%%%%%%%%%%%%%%%%%%%%%%%%%%%%%%%%
\\\vskip0.25in
%%%%%%%%%%%%%%%%%%%%%%%%%%%%%%%%%%%%%%%%%%%%%%%%%%%%%%%%%%%%
\end{center}
%%%%%%%%%%%%%%%%%%%%%%%%%%%%%%%%%%%%%%%%%%%%%%%%%%%%%%%%%%%%
% Section Heading
%%%%%%%%%%%%%%%%%%%%%%%%%%%%%%%%%%%%%%%%%%%%%%%%%%%%%%%%%%%%


%%%%%%%%%%%%%%%%%%%%%%%%%%%%%%%%%%%%%%%%%%%%%%%%%%%%%%%%%%%%
% Bullet Point & Numbered list - lists can also be nested as below
%%%%%%%%%%%%%%%%%%%%%%%%%%%%%%%%%%%%%%%%%%%%%%%%%%%%%%%%%%%%
\begin{enumerate}      % first begin-----------]
\item At time t = 0 a particle is represented by the wave function\\
\begin{center}
    

$\Psi(x)$ = \begin{cases}
A\frac{x}{a}, & 0 \leq x \leq a\\
A\frac{(b-x)}{(b-a)}, & a \leq x \leq b\\
0, & \text{otherwise}
\end{cases} \\\\
\end{center}
where A, \textit{a} and \textit{b} are constants.
\begin{enumerate}
\item (3.3 points) Normalize $\Psi$ (i.e., find A in terms of \textit{A}, \textit{a}, and \textit{b}).

\item (3.3 points) Where is the particle likely to be found at \textit{t} = 0?.
\item (3.4 points) What is the expectation value of x?.
\end{enumerate}


 

\item {\bf The following questions refer to stars in the Table below.}\\
\textit{Note: there may be multiple answers.}\\ 
\begin{tabular}{|c|c|c|c|c|c|c|cccccc|}\hline
Name & Mass & Luminosity & Lifetime & Temperature & Radius & Variable?\\\hline
$\delta$ Scu. & $2.0 M_{\bigodot}$ & & $5.0 \times 10^{8}$ years & &2.0 $R_{\bigodot}$ & Y\\ \hline


$\gamma$ Del. & $0.7 M_{\bigodot}$ & & $4.5 \times 10^{8}$ years &5000K& &N\\ \hline


$\beta$ Cyg. & $1.3 M_{\bigodot}$ &3.5 $L_{\bigodot}$ &  & & & Y\\ \hline


$\eta$ Car. & $60. M_{\bigodot}$ &$10^{6} L_{\bigodot}$ & $8.0 \times 10^{5}$ years & & & Y\\ \hline


$\epsilon$ Eri.& $6.0 M_{\bigodot}$ &$10^{3} L_{\bigodot}$ & & 20,000 K &  & N\\ \hline

$\alpha$ Cen. & $1.0 M_{\bigodot}$ & & & 6000 K &1.0 $R_{\bigodot}$ & N\\ \hline
\end{tabular}\vskip 0.2in
\begin{enumerate}
\item (4 points) Which of these stars will produce a planetary nebula
\item (4 points) Elements heavier than \texttt{Carbon} will be produced in which stars.
\end{enumerate}
\item An electron is found to be in the spin state (in the \textit{z}-basis): $x = A \dbinom{3i}{4}$
\begin{enumerate}
\item (5 points) Determine the possible values of \textit{A} such that the state is normalized.
\item (5 points) Find the expectation values of the operators \textcolor{red}{$S_x$}, \textcolor{purple} {$S_y$}, \textcolor {Orange}{$S_z$}, and $\vec{S}^2$.
\end {enumerate}

The matrix representations in the \textit{z}-basis for the components of the electron spin operators are given by:\\

\color{red}{${\mathbf{{S_x} = }} \text{ } \frac{\bar{h}}{2}$
\begin{pmatrix} 
0 & 1 \\ 1 & 0
\end{pmatrix};}
& 
\color{purple}{${\mathbf{\phantom{asdfasdfasdf}{S_y} = }} \text{ } \frac{\bar{h}}{2}$
\begin{pmatrix} 
0 & -i \\ i & 0
\end{pmatrix};}
\color{orange}{${\mathbf{{\phantom{asdfasdfasdf}S_z} = }} \text{ } \frac{\bar{h}}{2}$
\begin{pmatrix} 
1 & 0 \\ 0 & -1
\end{pmatrix};}

\end {enumerate}



\\\vskip0.25in
Latex Example
\end{document}
